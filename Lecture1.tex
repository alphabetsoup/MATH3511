\documentclass{article}
%%%%%%%%%%%%%%%%%%%%%%%%%%%%%%%%%%%%%%%%%%%%%%%%%%%%%%%%%%%%%
% Lecture Specific Information to Fill Out
%%%%%%%%%%%%%%%%%%%%%%%%%%%%%%%%%%%%%%%%%%%%%%%%%%%%%%%%%%%%%
\newcommand{\LectureTitle}{Lecture \#1 Notes}
%\newcommand{\LectureDate}{\today}
\newcommand{\LectureDate}{July\ 30,\ 2013}
\newcommand{\LectureClassName}{MATH\ 3511}
\newcommand{\LatexerName}{Professor\ James\ Franklin}
%%%%%%%%%%%%%%%%%%%%%%%%%%%%%%%%%%%%%%%%%%%%%%%%%%%%%%%%%%%%%

% Change "article" to "report" to get rid of page number on title page
\usepackage{amsmath,amsfonts,amsthm,amssymb}
\usepackage{setspace}
\usepackage{Tabbing}
\usepackage{fancyhdr}
\usepackage{lastpage}
\usepackage{extramarks}
\usepackage{chngpage}
\usepackage{soul,color}
\usepackage{graphicx,float,wrapfig}
\usepackage{afterpage}
\usepackage{abstract}

% recommended from http://poisson.phc.unipi.it/~maggiolo/index.php/2008/12/latex-class-for-lecture-notes/
\usepackage[pdf]{pstricks}
\usepackage{hyperref}
\usepackage{mathtools}
\usepackage{booktabs}
\usepackage{multirow}
\usepackage{fancyhdr}
\usepackage{mparhack}
\usepackage{tikz}
\usepackage{mathdots}
\usepackage{xfrac}
\usepackage{faktor}
\usepackage{cancel}

% In case you need to adjust margins:
\topmargin=-0.45in
\evensidemargin=0in
\oddsidemargin=0in
\textwidth=6.5in
\textheight=9.0in
\headsep=0.25in

% Setup the header and footer
\pagestyle{fancy}
\lhead{\LatexerName}
\chead{\LectureClassName: \LectureTitle}
\rhead{\LectureDate}
\lfoot{\lastxmark}
\cfoot{}
\rfoot{Page\ \thepage\ of\ \pageref{LastPage}}
\renewcommand\headrulewidth{0.4pt}
\renewcommand\footrulewidth{0.4pt}

%%%%%%%%%%%%%%%%%%%%%%%%%%%%%%%%%%%%%%%%%%%%%%%%%%%%%%%%%%%%%
% Some tools
\newcommand{\enterTopicHeader}[1]{\nobreak\extramarks{#1}{#1 continued on next page\ldots}\nobreak
                                    \nobreak\extramarks{#1 (continued)}{#1 continued on next page\ldots}\nobreak}
\newcommand{\exitTopicHeader}[1]{\nobreak\extramarks{#1 (continued)}{#1 continued on next page\ldots}\nobreak
                                   \nobreak\extramarks{#1}{}\nobreak}

\newlength{\labelLength}
\newcommand{\labelAnswer}[2]
  {\settowidth{\labelLength}{#1}
   \addtolength{\labelLength}{0.25in}
   \changetext{}{-\labelLength}{}{}{}
   \noindent\fbox{\begin{minipage}[c]{\columnwidth}#2\end{minipage}}
   \marginpar{\fbox{#1}}

   % We put the blank space above in order to make sure this
   % \marginpar gets correctly placed.
   \changetext{}{+\labelLength}{}{}{}}

\setcounter{secnumdepth}{0}
\newcommand{\TopicName}{}
\newcounter{TopicCounter}
\newenvironment{Topic}[1][Problem \arabic{TopicCounter}]
  {\stepcounter{TopicCounter}
   \renewcommand{\TopicName}{#1}
   \section{\TopicName}
   \enterTopicHeader{\TopicName}}
  {\exitTopicHeader{\TopicName}}
  
\setcounter{secnumdepth}{0}
\newcommand{\ExampleSectionName}{}
\newcounter{ExampleSectionCounter}[TopicCounter]
\newenvironment{ExampleSection}[1][Example \arabic{ExampleSectionCounter}]
  {\stepcounter{ExampleSectionCounter}
   \renewcommand{\ExampleSectionName}{#1}
   \section{\ExampleSectionName}
   \enterTopicHeader{\ExampleSectionName}}
  {\exitTopicHeader{\ExampleSectionName}}

\setcounter{secnumdepth}{0}
\newcounter{ExampleBoxCounter}[TopicCounter]
\newcommand{\examplebox}[1]
  {
  % We put this space here to make sure we're disconnected from the previous
   % passage
   \stepcounter{ExampleBoxCounter}
   \noindent\fbox{\begin{minipage}[c]{\columnwidth}#1\end{minipage}}\enterTopicHeader{\ExampleSectionName}\exitTopicHeader{\ExampleSectionName}\marginpar{\fbox{\#\arabic{ExampleBoxCounter}}}
   % We put the blank space above in order to make sure this
   % \marginpar gets correctly placed.
   \vskip10pt
   }

\renewcommand{\contentsname}{{\normalsize Topics Covered}}
\renewcommand{\abstractname}{\LectureTitle\ Summary}
\renewcommand{\absnamepos}{flushleft}

%%%%%%%%%%%%%%%%%%%%%%%%%%%%%%%%%%%%%%%%%%%%%%%%%%%%%%%%%%%%%

\begin{document}
\begin{spacing}{1.1}
\newpage

\begin{abstract}
Prepared by Laurence Davies
\end{abstract}

\tableofcontents
\addtocontents{toc}{~\hfill\textbf{Page}\par}
\vskip10pt
\hrule
\vskip10pt

% When topics are long, it may be desirable to put a \newpage or a
% \clearpage before each Topic environment
%\newpage
\begin{Topic}[Introduction \Roman{TopicCounter}]
There are three main parts to this course:

\begin{enumerate}
\item Geometry (extra Euclidean geometry)
  \begin{itemize}
    \item Centres of triangles (mean, circumcentre, and so on)
    \item Circles
  \end{itemize}
\item Transformations in geometry 
  \begin{itemize}
    \item (Rotations, reflections, glide reflections, similarities) 
  \end{itemize}
\item Groups (abstract algebra) 
  \begin{itemize}
    \item e.g. Groups of symmetries:\\
%          \examplebox{

Consider the reflective symmetries of an equilateral triangle.\\ \\
% Generated with LaTeXDraw 2.0.8
% Tue Jul 30 12:28:22 EST 2013
% \usepackage[usenames,dvipsnames]{pstricks}
% \usepackage{epsfig}
% \usepackage{pst-grad} % For gradients
% \usepackage{pst-plot} % For axes
\scalebox{1} % Change this value to rescale the drawing.
{
\begin{pspicture}(0,-3.69)(10.03,3.79)
\pstriangle[linewidth=0.04,dimen=outer](5.42,-1.39)(5.08,3.98)
\psline[linewidth=0.04cm,linestyle=dashed,dash=0.16cm 0.16cm](8.98,-1.95)(2.82,1.47)
\psline[linewidth=0.04cm,linestyle=dashed,dash=0.16cm 0.16cm](2.0,-1.83)(7.88,1.35)
\psline[linewidth=0.04cm,linestyle=dashed,dash=0.16cm 0.16cm](5.44,-2.57)(5.42,3.77)
\psdots[dotsize=0.12](5.46,0.01)
\psarc[linewidth=0.04,arrowsize=0.05291667cm 2.0,arrowlength=1.4,arrowinset=0.4]{<-}(8.91,-1.92){0.35}{58.24052}{217.40535}
\psarc[linewidth=0.04,arrowsize=0.05291667cm 2.0,arrowlength=1.4,arrowinset=0.4]{<-}(2.33,-1.68){0.35}{304.11447}{117.474434}
\psarc[linewidth=0.04,arrowsize=0.05291667cm 2.0,arrowlength=1.4,arrowinset=0.4]{<-}(5.43,3.02){0.35}{357.27368}{182.48955}
\usefont{T1}{ptm}{m}{n}
\rput(7.76,-3.085){Reflective symmetries}
\pscustom[linewidth=0.01]
{
\newpath
\moveto(9.26,-2.85)
\lineto(9.28,-2.8)
\curveto(9.29,-2.775)(9.305,-2.705)(9.31,-2.66)
\curveto(9.315,-2.615)(9.32,-2.53)(9.32,-2.49)
\curveto(9.32,-2.45)(9.3,-2.375)(9.28,-2.34)
\curveto(9.26,-2.305)(9.22,-2.255)(9.2,-2.24)
\curveto(9.18,-2.225)(9.14,-2.205)(9.08,-2.19)
}
\pscustom[linewidth=0.01]
{
\newpath
\moveto(9.34,-2.85)
\lineto(9.47,-2.65)
\curveto(9.535,-2.55)(9.715,-2.25)(9.83,-2.05)
\curveto(9.945,-1.85)(10.025,-1.575)(9.99,-1.5)
\curveto(9.955,-1.425)(9.9,-1.305)(9.88,-1.26)
\curveto(9.86,-1.215)(9.825,-1.13)(9.81,-1.09)
\curveto(9.795,-1.05)(9.615,-0.73)(9.45,-0.45)
\curveto(9.285,-0.17)(8.975,0.32)(8.83,0.53)
\curveto(8.685,0.74)(8.51,1.025)(8.48,1.1)
\curveto(8.45,1.175)(8.33,1.435)(8.24,1.62)
\curveto(8.15,1.805)(8.005,2.03)(7.95,2.07)
\curveto(7.895,2.11)(7.6,2.28)(7.36,2.41)
\curveto(7.12,2.54)(6.765,2.715)(6.65,2.76)
\curveto(6.535,2.805)(6.385,2.855)(6.35,2.86)
\curveto(6.315,2.865)(6.265,2.89)(6.25,2.91)
\curveto(6.235,2.93)(6.21,2.95)(6.18,2.95)
}
\pscustom[linewidth=0.01]
{
\newpath
\moveto(9.42,-3.13)
\lineto(9.44,-3.16)
\curveto(9.45,-3.175)(9.46,-3.23)(9.46,-3.27)
\curveto(9.46,-3.31)(9.405,-3.38)(9.35,-3.41)
\curveto(9.295,-3.44)(9.17,-3.49)(9.1,-3.51)
\curveto(9.03,-3.53)(8.725,-3.55)(8.49,-3.55)
\curveto(8.255,-3.55)(7.92,-3.56)(7.82,-3.57)
\curveto(7.72,-3.58)(7.225,-3.61)(6.83,-3.63)
\curveto(6.435,-3.65)(5.9,-3.675)(5.76,-3.68)
\curveto(5.62,-3.685)(5.015,-3.655)(4.55,-3.62)
\curveto(4.085,-3.585)(3.52,-3.475)(3.42,-3.4)
\curveto(3.32,-3.325)(3.12,-3.155)(3.02,-3.06)
\curveto(2.92,-2.965)(2.8,-2.835)(2.78,-2.8)
\curveto(2.76,-2.765)(2.725,-2.68)(2.71,-2.63)
\curveto(2.695,-2.58)(2.67,-2.505)(2.66,-2.48)
\curveto(2.65,-2.455)(2.635,-2.405)(2.63,-2.38)
\curveto(2.625,-2.355)(2.62,-2.305)(2.62,-2.28)
\curveto(2.62,-2.255)(2.62,-2.205)(2.62,-2.18)
\curveto(2.62,-2.155)(2.62,-2.115)(2.62,-2.07)
}
\psarc[linewidth=0.04,arrowsize=0.05291667cm 2.0,arrowlength=1.4,arrowinset=0.4]{<-}(5.18,0.03){2.68}{94.87927}{200.6589}
\psarc[linewidth=0.04,arrowsize=0.05291667cm 2.0,arrowlength=1.4,arrowinset=0.4]{<-}(5.39,0.0){3.77}{338.8539}{200.0}
\usefont{T1}{ptm}{m}{n}
\rput(2.99,0.195){120}
\usefont{T1}{ptm}{m}{n}
\rput(1.71,1.875){240}
\usefont{T1}{ptm}{m}{n}
\rput(1.63,3.595){Rotational symmetries}
\end{pspicture} 
}


%}
    \item There will also be material on freeze groups and wallpaper groups.
  \end{itemize}
\end{enumerate}
\end{Topic}

\begin{Topic}[Historical Example: Lunes \Roman{TopicCounter}]
Please see the handout on Hippocrates Lunes (~450BC).
%\begin{ExampleSection}
Find a square with the same area as a curved lune.\\

% Generated with LaTeXDraw 2.0.8
% Tue Jul 30 12:48:41 EST 2013
% \usepackage[usenames,dvipsnames]{pstricks}
% \usepackage{epsfig}
% \usepackage{pst-grad} % For gradients
% \usepackage{pst-plot} % For axes
\scalebox{1} % Change this value to rescale the drawing.
{
\begin{pspicture}(0,4.2331495)(14.4952,10.2876005)
\pswedge[linewidth=0.04](8.475201,4.2676){6.0}{0.0}{180.0}
\pstriangle[linewidth=0.04,dimen=outer](8.475201,4.2576003)(12.0,6.0)
\psarc[linewidth=0.04](8.4852,-1.7823999){8.4852}{45.558964}{134.65895}
\psline[linewidth=0.04](14.426324,4.27579)(2.5210772,4.2531495)
\usefont{T1}{ptm}{m}{n}
\rput(8.4852,5.7226){C}
\usefont{T1}{ptm}{m}{n}
\rput(5.4752,8.3226){A}
\usefont{T1}{ptm}{m}{n}
\rput(11.5452,8.3426){B}
\usefont{T1}{ptm}{m}{n}
\rput(8.4552,8.1826){D}
\rput{-45.0}(-4.569971,8.922314){\psframe[linewidth=0.04,dimen=outer](8.6852,10.1776)(8.2852,9.7776)}
\usefont{T1}{ptm}{m}{n}
\rput(6.2252,7.7426){a}
\usefont{T1}{ptm}{m}{n}
\rput(10.5852,7.7626){b}
\usefont{T1}{ptm}{m}{n}
\rput(8.4652,4.5026){c}
\end{pspicture} 
}

Examine areas $A$, $B$, $C$ and $D$. This example will show that $C=A+B$ and hence the area of the lune $A+D+B=C+D$ which is the area of the triangle $\frac{1}{2}\times base\times height$.\\

Consider the Pythagorean relationship $c^2=a^2+b^2$.\\

The relationship between an area and the corresponding chord length is a quadrature 
i.e. $A = \lambda a^2$ where $\lambda$ is the same for all three segments.


%\end{ExampleSection}
\end{Topic}



%\newpage

\end{spacing}
\end{document}
